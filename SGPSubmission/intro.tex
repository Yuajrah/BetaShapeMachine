\section{Introduction}
\label{sec:intro}

\vspace{-2mm}

Discovering geometric and semantic relationships of 3D shapes is fundamental to several computer graphics applications. In particular, several tools for geometric modeling, manufacturing, shape retrieval and exploration can benefit from algorithms that automatically extract part, region and point correspondences within large shape collections. During the recent years, due to the growing number of online 3D shape repositories (Trimble Warehouse, Turbosquid and so on), a number of algorithms have been proposed to jointly analyze shapes in large collections to discover such correspondences. The key advantage of these algorithms is that they do not treat shapes in complete isolation from each other, but rather extract useful mappings and correlations between shapes to produce results that are closer to what a human would expect. 

Although there has been significant progress on analysis of shapes with similar structure (e.g., human bodies) and similar types of deformations (e.g, isometries or near-isometries), dealing with collections of structurally and geometrically diverse shape families (e.g., furniture, vehicles) still remains an open problem. A promising approach to handle such collections is to iteratively compute part or/and point correspondences and existence using part-based or local non-rigid alignment \cite{Kim13,Huang13}. The rationale for such approach is that part correspondences can help localize point correspondences and improve local alignment, point correspondences can refine segmentations and local alignment, and in turn more accurate local alignment can further refine segmentations and point correspondences. One of the drawbacks of existing methods following this approach is that strict point-to-point correspondences are not that suitable for collections of shapes whose parts exhibit significant geometric variability, such as airplanes, bikes and so on. In addition, these methods use templates made out of basic primitives (e.g., boxes in \cite{Kim13}) or other mediating shapes \cite{Huang13}, whose geometry may drastically differ from the surface geometry of several shapes in the collection leading to inaccurate correspondences. In the case of template fitting, approximate statistics on the used primitives can be used to penalize unlikely alignments (e.g., Gaussian distributions on box positions and scales). However, these statistics capture rather limited information about the actual surface variability in the shapes of the input collection. \rev{Similarly, in the context of shape synthesis, shape variability is often modeled with statistics over predefined part descriptors that cannot be directly mapped back to surfaces \cite{Kalogerakis12} or parameters of simple basic primitives, such as boxes \cite{Fish14,Averkiou14}}.

\rev{We present a method that analyzes and synthesizes 3D shape collections by learning representations of surface variability from scratch. Our method has two main components: a probabilistic deformation model and a generative  surface model. The probabilistic deformation model jointly estimates fuzzy point correspondences and part segmentations of shapes through learned part-based templates. In contrast to previous works that use simple geometric primitives or pre-existing shapes as templates, our method  learns the template geometry and deformations from the input collection. The deformation model provides input to our generative surface model that aims to learn geometric and structural relationships of parts and corresponding surface point positions. Following the concept of deep learning (or hierarchical learning) \cite{Bengio09} widely used in computer vision and natural language processing, the key idea of our generative model is to learn relationships in the surface data hierarchically: our model learns geometric arrangements of points within individual parts through a first layer of latent variables. Then it encodes  interactions between the latent variables of the first layer through a second layer of latent variables whose values correspond to relationships of surface arrangements across different parts. Subsequent layers mediate  higher-level relationships related to the overall shape geometry, semantic attributes and structure. The hierarchical architecture of our model is well-aligned with the compositional nature of shapes: shapes usually have a well-defined structure,      their structure is defined through parts, parts are made of patches and point arrangements with certain regularities.}
%The latent variables of our model may have certain interpretations: for example,  some latent variables of the first layer are associated with the shape of airplane wings or tailplanes (e.g., straight or delta-shaped wings). Variables of the second layer capture geometric relationships of airplane parts (e.g., swept wings usually co-occur with swept tailplanes). Variables of the third layer are related to  high-level airplane attributes (e.g. fighter jets or propeller aircraft types).

\rev{Our method  can jointly learn the probabilistic deformation model and the generative surface model leading to improved shape correspondence. In addition, the generative model can be sampled to generate plausible point-sampled surfaces that automatically drive shape synthesis. \ Finally, its uppermost layer produces a compact shape descriptor, or feature representation, that can be used to perform fine-grained classification (e.g., airplanes can be categorized to fighter jets, propeller aircraft,  unmanned aerial vehicles, and so on).}

%We note that our method is complementary to functional maps \cite{Ovsjanikov12,Huang14}. Our method infers probabilities over points and part correspondences, following the notion of fuzzy (soft) correspondences \cite{Kim12,Solomon12} that can be considered as special cases of functional maps. \rev{We believe that our method is also complementary to methods that employ cycle consistency constraints \cite{Huang12,Huang14} and human pose or functionality-related priors \cite{Kim14}.} These methods, including functional maps, can benefit from initial point or part correspondences that our method can provide. 

\textbf{Contributions.} The contribution of our work is two-fold. First, we provide a probabilistic deformation model that estimates fuzzy point and part correspondences within structurally and geometrically diverse shape families. The main difference with previous work is that our method learns the geometry and deformation parameters of templates within a fully probabilistic framework to optimally achieve these tasks instead of relying on fixed primitives or pre-existing shapes. \rev{Second, we introduce a deep-learned probabilistic generative model of 3D shape surfaces that can be used to further optimize shape correspondences, synthesize surface point arrangements, and produce  compact shape descriptors for fine-grained classification.  Both probabilistic models can be learned together leading to joint shape analysis and synthesis.} To the best of our knowledge, our method is the first to apply deep learning for training generative models of 3D\ shape surfaces. In contrast to previous work on generative probabilistic models that rely on highly structured databases (e.g., manually segmented shapes), our deep-learned model synthesizes shapes without or minimal human supervision. While previous generative models usually encode only high-level part and shape descriptors, our model directly encodes surface geometry and shape structure. Instead of mixing-and-matching parts from a database to synthesize shapes as frequently done in prior work, the sampled surface point arrangements of our model can be used to also deform the input collection parts to create more novel shape variations. 
